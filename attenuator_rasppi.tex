\documentclass{article}
\usepackage{circuitikz}
\usepackage{pdfpages}

\begin{document}

% Page 1: Schematic
\begin{titlepage}
    \centering
    \vspace*{1in}
    \textbf{\LARGE Schematic}
    \vspace{1.5in}
    
    \scriptsize % Adjust text size for schematic
    \begin{circuitikz}
        % Nodes
        \node at (0,0) (pi) {};
        \node at (2,0) (res1) [circ] {};
        \node at (4,0) (cap) [circ] {};
        \node at (6,0) (res2) [circ] {};
        \node at (8,0) (ant) {};

        % Components
        \draw (pi) node[left]{Raspberry Pi GPIO14 (UART TX)} -- (res1) to[R, l_=$R_1$] (cap);
        \draw (cap) to[C, l_=$C$] (res2);
        \draw (res2) to[R, l_=$R_2$] (ant);
        \draw (ant) node[right]{Antenna (50 cm, SMA Connector)};

        % Ground
        \node at (4,-2) (gnd) {Ground};
        \draw (cap) -- (4,-1.5) -- (gnd);
    \end{circuitikz}
\end{titlepage}

% Page 2: Table
\begin{titlepage}
    \centering
    \vspace*{1in}
    \textbf{\LARGE Component Table}
    \vspace{1.5in}
    
    \begin{tabular}{|c|c|c|}
        \hline
        Symbol & Description & Value/Formulas \\
        \hline
        $R_1$ & Resistor 1 & 10 k$\Omega$ \\
        $R_2$ & Resistor 2 & 1 k$\Omega$ \\
        $C$ & Capacitor & 10 nF \\
        \hline
    \end{tabular}
\end{titlepage}

\end{document}
